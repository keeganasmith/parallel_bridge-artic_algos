\documentclass[conference]{IEEEtran}

% Packages
\usepackage{cite}
\usepackage{amsmath,amssymb,amsfonts}
\usepackage{algorithmic}
\usepackage{graphicx}
\usepackage{textcomp}
\usepackage{xcolor}

% Title
\title{Your Paper Title Here}

% Authors
\author{
\IEEEauthorblockN{Keegan Smith}
\IEEEauthorblockA{
\textit{Computer Science} \\
\textit{Texas A\&M University} \\
College Station TX, United States \\
keeganasmith2003@tamu.edu
}
}

% Begin Document
\begin{document}

\maketitle

\begin{abstract}
Bridges have important applications in areas like networking, cybersecurity, and biological sciences. In this paper we propose a parallel bridge-finding algorithm which leverages union-find. Our implementation scales to at least 192 cores on a single node. 
\end{abstract}

\begin{IEEEkeywords}
Parallel Algorithms, Bridges, Graphs
\end{IEEEkeywords}

\section{Introduction}
Graphs are used in many fields of science to model systems, thus being able to analyze these graphs quickly and efficiently is important. \\
In this paper we propose a parallel algorithm for finding bridges in a graph which we show scales to hundreds of cores. \\
To the best of our knowledge, no other works have shown evidence of a scalable parallel bridge finding algorithm.
\section{Related Work}
The world of parallel bridge algorithms is somewhat sparse. Kumar and Singh ~\cite{kumar2021efficient} proposed a parallel bridge finding algorithm focused on DFS, however their implementation only seemed to scale up to 50 cores.
\section{Methodology}
Our algorithm uses union find to construct a spanning tree of the graph. Any edge which is not included in the spanning tree is guaranteed not to be a bridge. We construct a spanning tree again, however this time we add the not bridge edges to the union first. We repeatedly construct spanning trees until 
\section{Results}
Present your findings and analysis.

\section{Discussion}
Interpretation of your results.

\section{Conclusion}
Summary and possible future directions.

\section*{Acknowledgment}
(Optional) Acknowledge support, grants, or contributions.

\bibliographystyle{IEEEtran}
\bibliography{references}
\end{document}
